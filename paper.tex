\documentclass[conference]{IEEEtran}
	\pdfpagewidth=8.5truein
	\pdfpageheight=11truein

\usepackage{graphicx}
\usepackage{subfig}
\usepackage[bookmarks=false]{hyperref}
\usepackage{xspace}
\usepackage{listings}
\usepackage[usenames, dvipsnames]{color}
\usepackage{amssymb}



\usepackage{booktabs}         

\graphicspath{{./img/}}

%\hyphenation{op-tical net-works semi-conduc-tor}

%
\def\sharedaffiliation{%
\end{tabular}
\begin{tabular}{c}}
%

\lstset{language=Python}

\definecolor{LightGray}{RGB}{250,250,250}

\lstdefinestyle{custompython}{
	captionpos=b,                    % sets the caption-position to bottom
	% frame=tb,
	xleftmargin=\parindent,
	language=Python,
	basicstyle=\footnotesize\ttfamily,
	keywordstyle=\bfseries\color{MidnightBlue},
	morekeywords={*,python,times,test_folder,url},
	stringstyle=\color{PineGreen},
  	commentstyle=\color{Magenta},
  	backgroundcolor=\color{LightGray}
}


\newcommand{\tool}{Flask Dashboard\xspace}
\newcommand{\zee}{Zeeguu\xspace}
\newcommand{\zeeurl}{\xspace}
\newcommand{\git}{\texttt{git}\xspace}
\newcommand{\install}{{\small \texttt{pip install flask\_monitoring\_dashboard}}\xspace}
\newcommand{\activeUserCount}{two hundred\xspace}
\newcommand{\code}[1]{\texttt{#1}\xspace}
\newcommand{\perspective}[1]{{\bf {\small {\texttt{#1}}\xspace}}}
\newcommand{\withheld}{withheld due to double blind}
\newcommand{\citationwithheld}{{\em [citation withheld due to double blind]}}


\newcommand{\lesson}[1]{{\bf {\small {\texttt{Lesson: }}\xspace}}: #1}

\usepackage{fourier-orns}

\definecolor{myred}{RGB}{230, 20, 70}
\definecolor{mygreen}{RGB}{60, 180, 75}
\definecolor{uploadUserData}{RGB}{135, 143, 197}
\definecolor{feedItems}{RGB}{245, 130, 48}
\definecolor{topArticles}{RGB}{139, 210, 121}
\definecolor{bookmarksToStudy}{RGB}{210,245, 60}


\newcommand{\niceseparator}
	{
		\begin{center}
  		% $\ast$~$\ast$~$\ast$
  		% $\clubsuit$~$\clubsuit$~$\clubsuit$
  		\leafleft
		\end{center}
	}

% Endpoint Names 
%\newcommand{\epDecoration}[1]{{\small {\bf #1}}\xspace}
\newcommand{\epDecoration}[1]{\code{\small #1}}

\newcommand{
			\epColorDecoration}[2]{#2\code{\small(\color{#1}{\small$\blacksquare$}\color{black})}
		}
				%\xspace}

\newcommand{\epTranslations}{\epDecoration{get\_possible\_translations}}
\newcommand{\epTranslationsColor}{\epColorDecoration{myred}{\epTranslations}}

\newcommand{\epOutcome}{\epDecoration{report\_exercise\_outcome}}
\newcommand{\epOutcomeColor}{\epColorDecoration{mygreen}{\epOutcome}}

\newcommand{\epUserActivity}{\epDecoration{upload\_user\_activity\_data}}
\newcommand{\epUserActivityColor}{\epColorDecoration{uploadUserData}{\epUserActivity}}

\newcommand{\epFeedItems}{\epDecoration{get\_feed\_items\_with\_metrics}}
\newcommand{\epFeedItemsColor}{\epColorDecoration{feedItems}{\epFeedItems}}

\newcommand{\epTopArticles}{\epDecoration{top\_recommended\_articles}}
\newcommand{\epTopArticlesColor}{\epColorDecoration{topArticles}{\epTopArticles}}

\newcommand{\epBookmarksToStudy}{\epDecoration{bookmarks\_to\_study}}
\newcommand{\epBookmarksToStudyColor}{\epColorDecoration{bookmarksToStudy}{\epBookmarksToStudy}}


% Author Comments / Discussion

\definecolor{mlcolor}{RGB}{140, 140, 205}
\definecolor{vacolor}{RGB}{255, 0, 255}

\newcommand{\ml}[1]{ 
	{\footnotesize \color{mlcolor}ML: #1}
	}

\newcommand{\ins}[1]{ 
	{\color{blue}#1}
	}


\newcommand{\va}[1]{ 
	{\footnotesize \color{vacolor}VA: #1}
}


\newcommand{\mltp}[1]{\ml{Thijs, Patrick: #1}}
\newcommand{\mlv}[1]{\ml{Vasilios: #1}}

\definecolor{todocolor}{RGB}{200, 140, 200}

\newcommand{\todo}[1]{ 
	{\bfseries \color{todocolor}Todo: #1}
	}

\newcommand{\Fref}[1]{Fig.~\ref{#1}}
\newcommand{\Sref}[1]{Sec.~\ref{#1}}




\begin{document}
	

% An Agile Analytics Platform for Visualizing Evolving Flask-Based Python Web Services
% A Dashboard for the Agile Monitoring of Evolving Web Services: For the Little Man
% A Dashboard for Monitoring Flask \& Python-Based Web Service Evolution
% Monitoring the Evolution of Flask \& Python-Based Web Services
% A Dashboard for Monitoring the Evolution of Flask \& Python-Based Web Services
% Continuously Monitoring the Evolution of Flask \& Python-Based Web Services
% A Dashboard for Continuously Monitoring the Evolution of Flask \& Python-Based Web Services
\title{Continuously Monitoring the Evolution of Flask \& Python-Based Web Services}

\author{
\IEEEauthorblockN{Author One, Segundo Autor, Troisi\`eme Auteur}\\
Famous Institute for Software Engineering and Computer Science\\
University of Great City, Some Country\\
Email: \{first.last\}@university.to 
}


% make the title area
\maketitle

%!TEX root=paper.tex

\begin{abstract}

  Python is one of the fastest growing programming languages of the moment. Flask, a Python-based web framework is the technology used by tens of thousands of web applications. These applications, from interactive websites to service APIs, lack a technology specific service monitoring solution. 

  In this paper, we present \tool, a tool that provide insight into the utilization and performance of evolving Flask-based web services. We present the ease with which the tool can be integrated in an already existing web application, discuss some of the visualization perspectives that the library provides and point to some future challenges for similar libraries.

\end{abstract}




\section{Introduction}


% \todo{motivate with: monitoring in a world which is becoming more and more service-oriented is essential because it allows to perform three main types of actions: system adaptations to provide the request service at the desired level of quality, enabling flexibility in dealing with changing requirements and modes of operation, and enabling operational awareness through dashboards organizing the collected data~\cite{pernici2016monitoring}. In this work we focus on the last part and we discuss how a minimal effort probe-based solution for a particular type of web services can be used to facilitate the other two types of actions by involving the developer.}

% \todo{explain: in~\cite{vogel2017low} we introduced the \tool an a high level with a focus on presenting its performance visualization aspects; in this work we only summarize these features by means of introducing some of the provided functionalities of the \tool. We focus on discussing how the tool is implemented and operating, provide a deeper look at its capabilities for integration with version control and continuous integration environments, introduce a mechanism for performance prediction, and provide an evaluation of the proposed approach based on a case study. }

%Every system is a distributed system nowadays \cite{cavage2013there}. Indeed a very large number of applications and web applications are nowadays implemented as two-tier architectures with a front-end implemented with web technologies and a service back-end.
%\ml{I'm not completely happy with this paragraph}
{\em There is no getting around it: you are building a distributed system} argues Cavage  in a recent article written for the {\em Communications of the ACM} \cite{cavage2013there}. Indeed, even the simplest second-year student project is a web application implemented as two-tier architecture with a Javascript/HTML5 front-end a service backend, usually a REST API.

% \hfill mds
% Many contemporary programming languages are offering libraries, modules, or frameworks that facilitate the development of such architectures. 
Python is one of the most popular programming language choices for implementing the back-end of web applications. GitHub contains more than 500K open source Python projects and the Tiobe Index\footnote{TIOBE programming community index is a measure of popularity of programming languages, created and maintained by the TIOBE Company based in Eindhoven, the Netherlands} ranks Python as the 4th most popular programming language as of June 2016. An analysis of  StackOverflow from September 2017 argues that {\em ``Python has a solid claim to being the fastest-growing major programming language''}\footnote{https://stackoverflow.blog/2017/09/06/incredible-growth-python/}.
 
% possible flask summary
Within the Python community, Flask\footnote{\url{http://flask.pocoo.org/}} is a very popular web framework\footnote{More than 25K projects on GitHub (5\% of all Python projects) are implemented with Flask (cf. a GitHub search for ``language:Python Flask'')}. It provides simplicity and flexibility by implementing a bare-minimum web server, and thus advertises as a micro-framework. The Flask tutorial shows how setting up a simple Flask {\em ``Hello World''} web-service requires no more than 5 lines of Python code \cite{ flask:tutorial}.
% end of summary
 
Despite their popularity, to the best of our knowledge, there is no simple solution for monitoring the evolving performance of Flask web applications. Thus, every one of the developers of these projects faces one of the following options when confronted with the need of gathering insight into the runtime behavior of their implemented services: 

  \begin{enumerate}

    \item Use a commercial monitoring tool which treats the subject API as a black-box (e.g. Pingdom, Runscope). 
    % , Graphite+Graphana+statd etc.

    \item Implement their own ad-hoc analytics solution, having to reinvent basic visualization and interaction strategies. 

    \item Live without analytics insight into their services.

  \end{enumerate}

%\todo{For the first point in the list, we can also argue that analytics solutions like Google Analytics can be used, but they have no notion of versioning/integration with the development life cycle. Feel free to cite \cite{papazoglou2011managing} for service evolution purposes}

For projects on a budget (e.g. research, startups) the first and the second options are often not available due to time and financial constraints. Even when using 3rd-party analytics solutions, a critical insight into the evolution of the exposed services of the web application, is missing because such solutions have no notion of versioning and no integration with the development life cycle.~\cite{papazoglou2011managing}

To avoid projects ending up in the third situation, that of living without analytics, in this paper we present \tool~ --- a low-effort service monitoring library for Flask-based Python web services that is easy to integrate and enables the {\em agile assessment} of service evolution. \cite{Nier12b}

As a case study, on which we will illustrate our solution, we are going to use an open source API which, for several years, was in the third of the above-presented situations.

% In the next section, we will present a case study of an open source research API which was for a long time in the third situation presented above -- deployed without analytics insight.

\subsection*{Extending Previous Work}

This paper is an extension of [Anonymous Paper by Anonymous Auhthors] which was published as a NIER paper in [Anonymous Conference]. The current paper extends that paper in the following ways: 
\begin{itemize}
  \item Providing more details about the case study in Section \ref{sec:case} and extending the data collection period with half year
  \item Novel perspectives and a better categorization of the perspectives presented in Sections \ref{sec:utilandperf} -- \ref{sec:evolution}
  \item A detailed discussion on automated outlier detection and monitoring (Section \ref{sec:outliers})
  \item An advanced discussion integration continuous integration environments in order to triangulate the live collected data with information from testing the endpoints with a constant load and introduce a mechanism for performance prediction (Section \ref{sec:testing})
  \item A more extended discussion about lessons learned while developing the tool including measurements on the overhead incurred by the presented tool \ml{(... if we make it... )}
\end{itemize}



\section{Case Study}

% \subsection{An Ecosystem With a Service API at its Core}
\label{sec:api}
\label{sec:case}

  \zee\footnote{\url{https://github.com/zeeguu-ecosystem/}} is a platform and an ecosystem of applications for accelerating vocabulary acquisition in a foreign language \cite{Lungu16}. 
%
  The architecture of the ecosystem has at its core an API and a series of ecosystem applications that together offer to a learner three main inter-dependent features:

  \begin{enumerate}

    \item \textit{Reader applications} that provide one-click (or one touch, on touch-enabled devices) translations for the words and expressions in a text that the reader does not understand. Besides facilitating reading, the translations serve as input to a user knowledge model that is used further to generate personalized exercises and further reading recommendations.

    \item \textit{Interactive exercises} are generated based on the texts that the reader has studied in the past and are scheduled in such a way as to optimize the memory retention and to prioritize the most important unknown vocabulary items.

    \item \textit{Article recommendations} are also personalized for the interests and language capabilities of each learner. The recommendations come with a difficulty estimation that helps the learner find articles with appropriate difficulty.

  \end{enumerate}

  \Fref{fig:zeeguuarch} presents a high-level view of the case study ecosystem.  The core API is implemented with Flask and Python provides and correspondingly three types of functionality: contextual translations, personalized exercise suggestions, and article recommendations. In total, the API provides a bit under 50 endpoints, out of which around a dozen are very frequently used. The development of the core API itself is a research project. One of the authors of this paper is part of the developer team that maintains and evolves the \zee API. 


    \begin{figure}[h!]
      \centering
      \includegraphics[width=0.7\linewidth]{zeeguu-architecture}
      \caption{The architecture of the \zee ecosystem}
      \label{fig:zeeguuarch}
    \end{figure}  


  At the time of writing, the ecosystem consists of a reader web application, a web-based exercises platform, and a smartwatch application, which are used by more than \activeUserCount active users on a regular basis. The users come from several Dutch highschools and a language center associated with a Dutch University. Several users are using it on their own, outside of any formal educational context. The learners use \zee to study a variety of of foreign languages including German, Dutch, French. They have also diverse native languages, including: Chinese, Dutch, and English. The system can scale to many languages by delegating the machine translation tasks to third party APIs. 



\subsection*{Study characteristics}

  We will use this \zee service API as a case study for this paper. 
  The API has been deployed for several years without any monitoring
  solution until June 2017, when the \tool has been deployed with it. 
  During the ten month deployment (between June 2017 and the time of writing this
  article, April 2018) more than 112.000 requests
  to the API have been recorded by the API \footnote{Note that the 
  recorded requests are a lower bound on the usage of the API, 
  as only a subset of the API endpoints have been actually instrumented} . The highest load observed 
  until now on the API consisted of 12K requests in a single day. 
  With the help of the applications in the ecoystem, in the aforementioned
  time period the users have 
  performed about 30K vocabulary exercises and have translated 
  about 28K words and experssions while reading foreign language 
  texts. 

  All the figures and measurements in this paper are captured from the actual deployment of \tool in the context of the \zee API. The figures are interactive offering basic data exploration capabilities: filter, zoom, and details on demand\cite{Shne99a}. The \tool deployment for the case study can be accessed publicly\footnote{\url{https://zeeguu.unibe.ch/api/dashboard}; username: {\em guest}, password: {\em icsme}}. 


 % \todo{create new account, update URL if necessary, update activeUserCount, add a couple of sentences of the current state of the case study with the college/language center}
% \ml{we should consider adding also one section in which the architecture/implementation and main features of the dashboard are presented before going on with discussing them in more depth in the following sections --- this should include a rundown on which views are provided from where (overview or per endpoint)}



\subsection{The Technology Stack}
\label{sec:flask}

 As discussed in the previous section, Flask is a microframework for Python. 
 The ``micro'' in microframework means Flask aims to keep the core simple but extensible. 
 The framework does not make many decisions for the user, such as what database to use and
 those decisions that it does make, such as what templating engine to use, can be changed. 

 Flask is built on top of WSGI -- The Web Server Gateway Interface -- a calling convention for web servers to forward requests to web applications or frameworks written in Python. A Python web application based on WSGI has to have one central callable object that implements the actual application. In Flask this is an instance of the Flask class. Each Flask application has to create an instance of this class itself and pass it the name of the module. The following code snippet presents a simple ``Hello World'' web service written in Python with the help of the Flask framework. 

\begin{lstlisting}[style=custompython]

from flask import Flask
app = Flask(__name__)

@app.route('/')
def index():
    return 'Hello World!'

\end{lstlisting}






%!TEX root=restructured.tex


  \newpage
  \section{LOC \#1: Service Utilization and Performance}

  Since it is taylored for Flask applications, to deploy the \tool one simply needs to use one line of code\footnote{We don't count the import statement that enables that line...} to bind the dashboard to their Flask web application\footnote{\ins{ In this paper we present the integration with APIs written in Python and Flask hoping that this will not prevent the reader from seeing the more general idea; all the tools we show here for Flask can be applied to other API technologies (e.g. Django) by simply providing a few back-end adapters in the right places. }}:

  % caption=Configuring the \tool is straightforward,
  \begin{lstlisting}[style=custompython]
  import flask_monitoringdashboard as dashboard

  # app is main application object
  # every Flask web application has one
  dashboard.bind(app) 

  \end{lstlisting}


  During binding, the \tool will search for all endpoints defined in the target application 
    \ins{ 
      and add function wrappers around all the corresponding endpoint functions
    }
% 
  % In order to monitor an endpoint, the \tool creates a function wrapper for the API function that corresponds to the endpoint. This way, the wrapper will be executed whenever that API call is made before the actual function is called. The wrapper contains the code that takes care of monitoring an endpoint. 
% 

   \ins{
    The tool takes advantage of the fact that the monitored API already has a web presence, and makes available one extra endpoint (i.e. \code{/dashboard}), which serves the interactive visualization perspectives presented in the remainder of this paper. 
  }  
 The first perspective presents all the automatically discovered endpoints and lets the user select the ones that should be monitored. 
  \Fref{fig:sep} shows that the last access time of every endpoint is tracked irrespective of whether it is selected by the user to be monitored or not. 


    \begin{figure}[h!]
      \centering
      \includegraphics[width=\linewidth]{selecting_endpoints.png}
      \caption{Once connected to an API the Dashboard presents the endpoints that are available for monitoring}
      \label{fig:sep}
    \end{figure}

  \ins{
    We have decided for an opt-in approach to monitoring to avoid any the performance penalties incurred by the dashboard to affect performance sensitive endpoints. We discuss performance issues later. Also, we discuss later one situation in which opt-out might be desirable. 
  }

  \ins{

    One alternative to allowing the user to use annotations would be to let them to annotate the code. However, this pollutes the code, and prevents deploying two versions which would monitor different endpoints. 

  }

\niceseparator

  \ins
  {

      The remainder of this section presents several of the interactive
      visualizations that become available without any further configuration\footnote{We recommend obtaining a color version of this paper for better readability}.

  }


%!TEX root=paper.tex


\newpage
\subsection{Utilization}
\label{sec:util}

\ins{Knowing how third parties use one's API is difficult even in the case of static dependencies. In the case of services, there is no other way but monitoring service utilization. \tool introduces a series of perspectives on utilization. }

\ins{
  
  The most basic possible view shows the cummulative information about all the endpoints and the number of calls to that endpoint over the lifetime of its tracking as well as in the current day and the last seven days.  
}


  \begin{figure}[h!]
  \centering
  \includegraphics[width=\linewidth]{basicest-utilization}
  \caption{....}
  \label{fig:basicest}
  \end{figure}

This kind of information is already very useful, and can provide the API maintainer with insight into the evolution of their system. By looking at \Fref{fig:basicest}, which presents the top 7 endpoints (for lack of space) one can already see several patterns: 

\begin{itemize}

  \item {\bf Frequent Patterns of System Usage}. The two most used endpoints stand for the two main activities in the system: 

  \begin{itemize}

    \item \epTranslations is an indicator of the amount of foreign language reading the users are doing

    \epOutcome is an indicator of the amount of foreign vocabulary practice the users are doing.

  \end{itemize}

  \item {\bf Sudden Increase of Endpoint Usage}. One endpoint (\epUserActivity) has been disproportionately been used in the last day and last week; much more than before. 

  \item {\bf Possibly Discontinued Endpoint Usage}. One endpoint (\epFeedItems) has not been used in the last 7 days

\end{itemize}

\lesson{Although very useful, it was only after several months of using the system that the client requested the extra columsn for one day and seven days.}

  % \begin{figure}[h!]
  % \centering
  % \includegraphics[width=0.5\linewidth]{number_of_requests_}
  % \caption{The number of requests per endpoint per day view shows the overall utilization of the monitored application}
  % \label{fig:aeu}
  % \end{figure}




The table view is limited, and the one day and one week periods are conveniently but arbitrarily selected. For a more detailed evolution of utilization, Figure \ref{fig:aeu} shows the \perspective{Daily Utilization} perspective on endpoint utilization that \tool provides: a stacked bar chart of the number of hits to various endpoints grouped by day\footnote{Endpoint colors are the same in different views}. Figure~\ref{fig:aeu} in particular shows a peak utilization, a day when the API had more than 12.000 hits. 

\begin{figure}[!ht]
\centering
\includegraphics[width=\linewidth]{number_of_requests_}
\caption{Usage patterns become easy to spot in the requests per hour heatmap}
\label{fig:aeu}
\end{figure}


% \begin{figure}[h!]
%   \centering
%   \subfloat[Daily Utilization]{\includegraphics[width=.42\columnwidth]{number_of_requests_}\label{fig:aeu}}
%   \subfloat[Hourly Utilization]{\includegraphics[width=.5\columnwidth]{daily_patterns_}\label{fig:dp}}
%   \caption{Some of the available views\label{fig:views}}
% \end{figure}


% The way users interact with the platform can also be inferred since the endpoints are indicators of different activity types, e.g.: 



% Besides showing the overall utilization, this endpoint provides the maintainer with information relevant for decisions regarding endpoint deprecation --- one of the most elementary ways of {\em understanding the needs of the downstream}\cite{Haen14a}. In our case study, the maintainer realized that one endpoint which they thought was not being used (i.e. \code{words\_to\_study}), contrary to their expectations, was actually being used\footnote{A complementary type of usage information can also be discovered in the view presented in Figure \ref{fig:sep} where seeing that an endpoint is never accessed can increase the confidence of the maintainer that a given endpoint is not used, although it can never be used a proof.}.

% \niceseparator

%   \todo{Add the time series graph and discuss it before the heatmap? We can then sell the heatmap better} 
%   \ml{Not sure about which graph you refer to here V}

Another utilization perspective is the \perspective{Hourly Utilization} in which  the \tool can highlight {\em cyclic patterns of usage per hour of day} by means of a heatmap, as in \Fref{fig:dp}. 


\begin{figure}[!ht]
\centering
\includegraphics[width=\linewidth]{daily_patterns_}
\caption{Usage patterns become easy to spot in the requests per hour heatmap}
\label{fig:dp}
\end{figure}


Figure \ref{fig:dp} shows the API not being used during the early morning hours, with most of the activity focused around working hours and some light activity during the evening. This is consistent with the fact that the current users are all in the central European timezone. Also, the figure shows that the spike in utilization that was visible also in the previous graph happended in on afternoon/evening.








%!TEX root=paper.tex

\subsection{Performance}
\label{sec:perf}

Knowing the performance of one’s API is critical for the quality of a service API. Flask Dashboard introduces a series of perspectives on performance, which focus on the response times of the various endpoints.


\ins{HERE? ELSEWHERE? Actually, probably better for performance. The importance of the last day and last seven days became clear only after having used the dashboard for several months. }



The \tool also collects information regarding endpoint performance. The view in \Fref{fig:ep} summarizes the response times for various endpoints by using a box-and-whiskers plot. 


 \begin{figure}[!ht]
   \centering
   \includegraphics[width=0.85\linewidth]{endpoint_performance_}
   \caption{The response time (in ms) per monitored endpoint view allows for identifying performance variability and balancing issues}
   \label{fig:ep}
 \end{figure}



In the Zeeguu case study, one of the slowest endpoints, and one with the highest variability as shown in \Fref{fig:ep} is \epFeedItemsColor which retrieves a list of recommended articles for a given user. However, since a user can be subscribed to anything from one to three dozen article sources, and since the computation of the difficulty is personalized and it is slow, the variability in time among users is likely to be very large. 

From this view it became clear to the maintainer that four of the endpoints had very large variation in performance.   The most critical for the application and consequently the one optimized first was \epTranslationsColor endpoint which was part of an interactive loop in the reader applications that relied on the Zeeguu API. Moreover, cf. \Fref{fig:aeu} this endpoint is one of the most used in the system.



  \subsection{Automated Outlier Detection and Monitoring}
  
  When an API is called from within a highly interactive application (as it is the case with the case study in this paper) 
  of particular interest to the API developers are performance {\em outliers}. 

  Indeed, a translation request that takes three times more than expected can seriously decrease the perceived quality of the application. Thus, identifying, collecting all appropriate data, and diagnosing the root causes of such outliers is especially critical in improving the quality of an application. 
  
  % In the context of the RESTful services support by Flask, this means request serving times that deviate from the average to an unexpected degree. 
  
  For this purpose the \tool tracks for every monitored endpoint a {\em running average} response time value\footnote{\ins{For performance reasons, we assume that the response times for the endpoints are normally distributed. Otherwise, more general density distribution information must be collected in real time.}}. When it detects that a given request is an outlier with respect to this past average running value, it triggers the {\em outlier data collection routine} which stores \ins{extra information} about the current execution environment. A configurable threshold with a default value of $2.5$ times the running average response time is used for this purpose. 

  For every detected outlier request, the \tool collects information about the current Python stack trace, CPU load, memory consumption, request parameters, etc. in order to allow the maintainer to investigate the causes of these exceptionally slow response times. In this way it is possible to get \ins{detailed insight into the operation of the application in the extreme cases without unnecessarily burdening it with logging this information for every request}.


  \begin{figure}[h!]
    \centering
    \includegraphics[width=0.9\linewidth]{outlier-annotated}
    \caption{Automatically collected outlier information}
    \label{fig:figure1}
  \end{figure}
  

The bottom panel shows the stack trace. 
In this particular case, it is revealing for the developer to learn that at the time of the stack trace snapshot, the code was in the google\_translator.py: indeed, the system uses as back-end multiple translators, and it has been observed that many of the outliers happen to be waiting in the google translator.

This information has to be corroborated with the observations that neither the memory nor the processor are overloaded at the moment. Thus this functionality in microsoft\_translator is really slow in itself, and this is not a result of the machine being overloaded for example. 

\lesson{We need more advanced outlier analysis tools... Instead of manually searching in the page to build stats about Google vs. Microsoft}











%!TEX root=paper.tex

  \section{Automated Outlier Detection and Monitoring}
  \label{sec:outliers}
  % \va{Shouldn't this be promoted to a section? It is neither utilization- nor performance- related. Either that, or Section~\ref{sec:user} should be folded into this section too, and then Sections~\ref{sec:version} and~\ref{sec:regression} merged into another section. Otherwise we have a very scattered paper}
  
  When an API endpoint is called from within a highly interactive application (as it is the case with the case with \epTranslationsColor in the previous section) of particular interest to the API developers are performance {\em outliers}.   Indeed, a translation request that takes three times more than expected can seriously decrease the perceived quality of the application. Thus, identifying, collecting all appropriate data, and diagnosing the root causes of such outliers is especially critical in improving the quality of an application. 
  
  % In the context of the RESTful services support by Flask, this means request serving times that deviate from the average to an unexpected degree. 
  
  For this purpose the \tool tracks for every monitored endpoint a {\em running average} response time value\footnote{\ins{For performance reasons, we assume that the response times for the endpoints are normally distributed. Otherwise, more general density distribution information must be collected in real time.}}. When it detects that a given request is an outlier with respect to this past average running value, it triggers the {\em outlier data collection routine} which stores \ins{extra information} about the current execution environment. A configurable threshold with a default value of $2.5$ times the running average response time is used for this purpose. 

  For every detected outlier request, the \tool collects information about the current Python stack trace, CPU load, memory consumption, request parameters, etc. in order to allow the maintainer to investigate the causes of these exceptionally slow response times. In this way it is possible to get \ins{detailed insight into the operation of the application in the extreme cases without unnecessarily burdening it with logging this information for every request}.


  \begin{figure}[h!]
    \centering
    \includegraphics[width=0.9\linewidth]{outlier-annotated}
    \caption{Automatically collected outlier information}
    \label{fig:stack}
  \end{figure}
  

The bottom panel in \Fref{fig:stack} shows the stack trace. 
In this particular case, it is revealing for the developer to learn that at the time of the stack trace snapshot, the code was in the \code{google\_translator}: indeed, the system uses as back-end multiple translators, and it has been observed that many of the outliers happen while the system is waiting for the Google translator.
%
This information has to be corroborated with the observations that neither the memory nor the processor are overloaded at the moment. Thus this functionality in \code{microsoft\_translator} is really slow in itself, and this is not a result of the machine being overloaded for example. \va{last sentence does not parse}






%!TEX root=paper.tex

\section{Grouping Requests}
\label{sec:user}

For service endpoints which run computations in real time, the maintainer of a system might want to understand the endpoint performance on a per-user basis, especially for situations where the system response time is a function of some individual user load\footnote{E.g. in GMail some users have two emails while other have twenty thousand and this induces different response times for different users}.


To enable this, the \tool must be configured to associate an API call with a given user. The simplest way is to take advantage of the architecture of Flask applications in which a global \code{flask.request} object can be used to retrieve the session which can in turn lead to user identification: 

\begin{lstlisting}[style=custompython]  
# LOC #2: configure the dashboard
# to group requests by the user id
dashboard.config.group_by = 'User',
	lambda: Session.find(flask.request).user.id

\end{lstlisting}

Explain the code: the group by is assigned a tuple: 

\begin{itemize}
	\item 'User' stands for the name of the grouping
	\item the ``lambda'' stands for a function that takes as argument
	the global flask.request. Every application must have a way of
	associating a user with a request. Or an app with a request. 
	A user is usually retrieved from some session variable. 
	If we wanted to group by the user group for example, the code
	would change slightly by replacing ``user.id'' with ``user.group.id''
\end{itemize}

% \niceseparator

In Zeeguu, \epFeedItems retrieves a list of recommended articles for a given user. Cf. \Fref{fig:ep} it is the endpoint with the slowest response time and highest variability. The reason for this is that a user can be subscribed to anything from one to three dozen article sources and for each of the sources the system must compute the personalized difficulty of each article at every request. 


\begin{figure}[h!]
  \centering
  \includegraphics[width=.5\linewidth]{time_per_user}
  \caption{The \epFeedItems shows a very high variability across users}
  \label{fig:tpu}
\end{figure}


A \perspective{Per-User Performance} perspective should show the different response times for different users. Figure \ref{fig:tpu} presents a subset of the corresponding view in the \tool. The figure shows that the response times for this endpoint can vary considerably for different users with some extreme cases where a user has to wait a full minute until their recommended articles are shown\footnote{After seeing this perspective, the maintainer refactored the architecture of the system to move part the difficulty computation out of the interactive loop}.








%!TEX root=paper.tex
  
  \section{Version-Aware Monitoring}
  
  The Stack Overflow Developer Survey from 2018 revealed the fact that 88.4\% of professional developers use git for version control. 
  Given that versioning using git is the main way in which the source code of services evolves, it is natural to monitor their progress also across versions. 

  Version control in \tool can be supported in two ways,
  the developer explicitly states the current version, 
  or the current version is automatically detected based
  on some version control system. 


  % However, with the current configuration of the tool, it would be impossible for the maintainer to see the improvements resulting from the optimization. 

  If assume that the code that they run is deployed using .git, then with an extra line of configuration they can allow \tool to find the git\footnote{\url{https://git-scm.com/}} folder of the deployed service and automatically detect the version of the project that is running: 
    
\begin{lstlisting}[style=custompython]

# LOC #3: provide the dashboard with 
# information on where to find the 
# git information 
dashboard.config.git = 'path/to/.git'
  
      
\end{lstlisting}  
 
  If the \tool can automatically detect the current version of the project by reading the .git configuration as soon as the API is started and can then group measurements by version\footnote{Alternatively, the maintainer can add version identifiers manually for the web application through a configuration file if the system does not use git.}. 


  

  \subsection*{Evolving Utilization}

  \Fref{fig:mv-util} shows the \perspective{Multi-Version API Utilization} perspective which presents the utilization of the tracked endpoints across versions. Because the different versions might be deployed for very different periods of time, at the intersection of an endpoint and a version the chart does not plot the absolute utilization of that endpoint in that version but rather the percentage of all the API calls that go to that endpoint. Otherwise a version that is deployed for many weeks would make all those deployed for several days invisible.


    \begin{figure}[h!]
      \centering
      \includegraphics[width=0.9\linewidth]{utilization-evolution}
      \caption{The Evolution of All the API Endpoints Utilization Across System Versions}
      \label{fig:mv-util}
    \end{figure}

  With this view, several patterns are visible:
  \begin{itemize}
    
    \item \epFeedItems is discontinued a little bit after \epTopArticles is being introduced

  \end{itemize}


  \subsection*{Evolving Performance}

    \Fref{fig:tee} is a zoomed-in version of such a view for \epTranslations with versions increasing from top to bottom

    \begin{figure}[h!]
      \centering
      \includegraphics[width=0.9\linewidth]{translation_endpoint_evolution_}
      \caption{The Performance Evolution of the \epTranslations endpoint}
      \label{fig:tee}
    \end{figure}


  This view confirms that the performance of the translation endpoint improved in the recent versions: the median of the last three versions is constantly moving towards the left, and progresses from 1.4 seconds (in the top-most box plot in \Fref{fig:tee}) to 0.8 in the latest version (bottom-most box plot).


\subsection*{Evolving Groups}
  The limitation of the previous view is that it does not present the information also on a per version basis. To address this, a different visual perspective entitled \perspective{Multi-Version per-User Performance} can be defined. Figure \ref{fig:tuv} presents such a perspective by mapping the average execution time for a given user (lines) and given version (columns) on the area of the corresponding circle. 

\begin{figure}[h!]
  \centering
  \includegraphics[width=0.97\linewidth]{time_per_user_per_version}
  \caption{This perspective shows that the evolution of response times for individual users (horizontal lines) across versions (the x-axis) for a given endpoint}
  \label{fig:tuv}
\end{figure}


The colors represent users. The figure shows average performance varying  across users and versions with no clear trend: this is probably because varying user workload (i.e. number of sources to which the user is registered) is the reason for the variation in response times. \ins{one can see that for user 1 performance degrades over the versions. Given that for other users the performance does not degrade in the same way, it is probable that the problem might lay elsewhere: the server was overloaded or the endpoint is ``algorithmically slower''.}


  


%!TEX root=paper.tex


  \section{Performance Triangulation With Regression Monitoring}
  \label{sec:testing}

  Suppose the developer observes a performance degradation of a given API endpoint in the latest version of the system. How do they know whether the degradation is due to the performance of the code, the server load, or the workload mix of the user? 

  \vspace{0.1cm}

  
  One way of triangulating such an observation is by monitoring the evolution of the performance of the endpoints when tested with a constant load across versions. Thus, if the performance decreases in a new version, and the endpoint in question is tested with a constant load the problem must be the endpoint implementation. 

  \subsection*{Integrating with CI Servers}

  To monitor endpoint performance with a constant load \tool takes advantage of two best practices in software evolution: (1) an API must have unit tests for its endpoints, and (2) these unit tests will be run within a Continuous Integration (CI) server. By tracking the timing of these unit tests one can obtain a perspective on endpoint performance evolution with a constant load (as long as the unit tests do not change). 

  \tool can be configured to work together with CI frameworks like Travis\footnote{\url{https://travis-ci.org/}, one of the most popular CI solutions at the moment} that deal with automated integration testing. To do this one needs to simply add one extra line in the script that runs their continuous integration testing. In the case of Travis CI, a developer has to add the following line in the `.travis.yml' configuration file: 


  \begin{lstlisting}[style=custompython]  
# LOC #4: to be added to the .travis.yml file
python -m flask_monitoringdashboard.collect
   --test_folder=./tests_zeeguu_api --times=5 
   --url=https://zeeguu.unibe.ch/api/dashboard

  \end{lstlisting}

  Each time a new build is created through Travis, the \tool automatically detects all available unit tests defined by the application developer (in (\code{-{}-test\_folder)}) and iterates through each one of them a number of times (\code{-{}-times}) while monitoring the response times for each test. The resulting measurements are uploaded to a specific endpoint that is created by the tool at the API url (\code{-{}-url}). The data uploaded to this endpoint are persisted in a separate part of the \tool database so as not to contaminate the ``live'' monitoring data. 

      \begin{figure}[h!]
        \centering
        \includegraphics[width=0.85\columnwidth]{travis_builds-bw}
        \caption{Response times in 5 consequent Travis builds}
        \label{fig:builds}
      \end{figure}

  \Fref{fig:builds} is a screenshot from the dashboard showing the measured response times for 5 consequent builds, with 180 iterations of the unit tests executed in total across all endpoints. The outliers on the right of the figure are due to initial requests which must wait for a boot-up phase of the API.
  


  \subsection*{Preemptive Monitoring}
  The concept of {\em preemptive monitoring} of the application performance by means of instrumenting integration unit testing as the synthetic load is similar to the idea of augmenting service monitoring with online testing~\cite{metzger2010proactive}, i.e.~testing service-based applications by using dedicated test input in parallel to its normal use and operation. The difference is that we take advantage of the capability of the CI framework to create an emulated ``live'' environment for integration testing purposes, and use unit testing as the dedicated test input. % in order to measure performance. 
  
  While this integration testing environment is different from the production one, and the load used is purely synthetic, it can serve as an early performance indicator for the developer.  \Fref{fig:response_times_preemptive} shows the actual endpoint response times across five versions of the system (above) and the corresponding testing times for the synthetic load in the same five versions (below). 

      \begin{figure}[h!]
        \centering
        \includegraphics[width=0.75\columnwidth]{response_per_version_trunced_trend}


        \advance\leftskip-0.2cm
        \includegraphics[width=0.5\columnwidth]{travis_builds_no_outliers_trend}
        \caption{The reported response time (in ms) per deployed version in the observation period: 
        in the actual deployment (above) and using integration with Travis (below). 
        Green line hints to a correlation between the two sets of measurements. }        
        
        \label{fig:response_times_preemptive}
      \end{figure}

  Computing Pearson correlation ($r(3)=.93, p=.02$) between the median values of the two datasets (cf. Table \ref{tab:correlations}) shows a correlation between the two. Further investigation on the characteristics of this observation is currently ongoing and it will be elaborated further in future work.


    \begin{table}[h]
      
      \centering
      \begin{tabular}{lll}
        \toprule
        Iteration & \bfseries Live (median) & \bfseries Travis (median)\\
        \midrule
        1 & 1349.41 & 764.87\\ 
        2 & 1466.13 & 992.87\\
        3 & 1256.65 & 760.87\\
        4 & 1266.42 & 813.89\\
        5 & 1080.68 & 303.4\\
      \bottomrule
      
      \end{tabular}
      \caption{Median response times in Figure \ref{fig:response_times_preemptive}}
      \label{tab:correlations}
    \end{table}




  



%!TEX root=paper.tex
  
\newpage
\section{Overhead of the \tool}
\label{sec:overhead}

To measure the performance overhead of the \tool, we have implemented an automated benchmarking system. It is open source and available online and can be tested by the reader. The benchmark downloads the latest version of the \zee API and installs it in a Docker container. Then it calls several selected endpoints for 500 times each, tracking the response times. The endpoints are called in three different configurations: 

	\begin{enumerate}
		\item With no dashboard installed
		\item With the dashboard enabled but with the outlier detection deactivated
		\item With the dashboard enabled and the outlier treshold set to zero, thus effectively treating every request {\em like it were an outlier}\footnote{This forces all the requests to be treated as outliers, and thus provides insight into this situation, which otherwise would be hard to generate}.
	\end{enumerate}



\Fref{fig:bench} and Table \ref{tab:benchmark} present with violin plots and respectively descriptive statistics the distribution of times resulting from running the benchmark for three different endpoints on a quad-core machine, with Intel Core i5-4590 processor, @ 3.30GHz, 8G of RAM and 240GB SSD disk drive.


\begin{figure}[h!]
	\centering
	\includegraphics[width=\linewidth]{benchmark2.pdf}
	\caption{The distribution of response times when calling the three endpoints for 500 times in three conditions: no dashboard, dashboard but no outliers, dashboard and every request treated as an outlier}
	\label{fig:bench}
\end{figure}


%!TEX root=paper.tex

\newcommand{\yes}{YES}

\begin{table}[tb]
	
	\centering

	\begin{tabular}{rllrr}
		\toprule
		\bfseries Endpoint & \bfseries Dashboard & \bfseries Outliers & \bfseries Mean (ms) & \bfseries SD (ms) \\

		\midrule

	         available\_languages  &    &	    &   6.3 &  2.3 \\ 
	         available\_languages  &  \yes &    &  19.9 &  5.6 \\ 
	         available\_languages  &  \yes &  \yes &  29.4 &  5.4 \\ \\ 

	                user\_article  &    &	    &  13.9 &  2.6 \\
	                user\_article  &  \yes &    &  26.9 &  3.2 \\
	                user\_article  &  \yes &  \yes &  44.8 & 10.0 \\ \\

	    create\_default\_ex...     &    &	    &  76.0 & 22.4 \\ 
	    create\_default\_ex...     &  \yes &    & 313.6 & 43.2 \\
	    create\_default\_ex...     &  \yes &  \yes & 385.3 & 16.9 \\
	
		\bottomrule

	\end{tabular}
	\caption{Descriptive statistics of the response times reported by running the benchmark}
	\label{tab:benchmark}

\end{table}




	The data shows that the dashboard (without outliers) introduces for the faster endpoints an overhead of ~14ms and for the slower endpoint an overhead of ~40ms. 
	In the case of an outlier, the overhead is doubled, so the design decision of only collecting the outlier information in the exceptional situations is correct.

	Depending on the application this might be acceptable or this might be too much. In the case of the \zee case study, the maintainers find this acceptable. 



  

% %!TEX root=paper.tex

    \begin{figure*}[ht!]
      \centering
      \includegraphics[width=0.8\linewidth]{interceptor}
      \caption{The first thing that needs to be done, is to decorate the application object with an INTERCEPTOR}
      \label{fig:sep}
    \end{figure*}


\section{Architecture}

REMEMBER THE GOAL: Our main goal with the \tool issues to allow the integration of the dashboard with an API with minimal effort.

The fact that the \tool itself is being developed in Python using Flask makes binding to the services of a to-be monitored application developed with the same technologies easy and intuitive. However, by writing different backends that do the measurement and monitoring, the frontend can be reused. 

The viewpoints we showed are only a subset of all the viewpoints of the \tool. In practice, the tool provides other perspectives as long as they are a combination of: 

\begin{itemize}
  \item Cardinality: one endpoint vs multiple
  \item Objective: utilization vs. performance (response time)
  \item Evolution Axis: time vs. versions
  \item Grouping: Grouped (e.g. by user) vs. All
\end{itemize}






\ins{The first important question that such a 
service monitoring system should provide is 
information about what endpoints are being
used and by whom}. 

TODO: Find references, arguments, related work
that provide more details about why understanding
who uses endpoints, and which are used are important.

This is clearly the case in static analysis (see paper
by haenni and lungu...) but should even more be the
case in APIs.




  \newpage


  % This used to be in the first section... 
  % Maybe it fits better localized here: 
  Data collected by the wrappers are persisted in a local database. The SQLAlchemy Object Relational Mapper\footnote{\url{https://www.sqlalchemy.org/}} allows for a DB system independent solution. 

  By default the system is deployed with an SQLite database \footnote{\url{https://www.sqlite.org/}} is used for this purpose.

TODO: TALK ABOUT THE METAMODEL...


    \begin{figure}[ht!]
      \centering
        \includegraphics[width=0.8\linewidth]{db_schema}
        \caption{The model that supports the views presented in this paper}
        \label{fig:sep}
    \end{figure}

  As discussed in the introductory section, in previous work \withheld we introduced \tool, a drop-in Python library that allows developers to monitor their Flask-based Python web applications with minimal effort.
%
  The \tool is implemented for Python 3.6 and is available on the Python Package Index repository
  % \footnote{\url{https://pypi.python.org/pypi/flask-monitoring-dashboard/1.8}} 
  \footnote{\url{https://pypi.python.org/pypi/[\witheld]}} 
  from where it can be installed by running \install from the command line. 
%  
  The source code of the \tool is published under a permissive MIT license and is available on GitHub\footnote{\url{https://github.com/flask-dashboard}}.
  
  




%!TEX root=paper.tex

\section{Discussion}
\label{sec:discussion}

  % \todo{use this subsection to discuss limitations (from the previous evaluation section) and add the need to be able to handle the horizontal scaling of the application; remove sectioning and summarize briefly limitations that have already been discussed in the VISSOFT paper/move material to other sections; consider renaming as limitations or something similar}


  The main goal of the \tool design was to allow analytics to be collected and insight to be gained by making the smallest possible changes to a running API. 

  %In this paper we present the integration with APIs written in Python and Flask hoping that this will not prevent the reader from seeing the more general idea; all the tools we show here for Flask can be applied to other API technologies (e.g. Django) by simply providing a few back-end adapters in the right places.

  The presented approach was centered around an API written in Python and Flask, but, in principle, 
  we can use the same approach for other technologies. We showed here how to do it for Python and Flask because they are some of the most popular web development technologies at the moment. However,
  % we hope that this will not prevent the reader from seeing the more general idea. All 
  all the interactive perspectives we developed can be equally applied to other service technologies (e.g. Django) by simply providing a few back-end adapters in the right places (e.g. endpoint discovery, the \code{/dashboard} endpoint deployment, etc.)
  As such, our approach is more generic than the technologies that it relies upon for its implementation. 

  At the same time, the presented approach also comes with several limitations. Some of these limitations might also represent challenges for the future builders of similar tools.

  \subsection*{The Dashboard Overhead}

    We have provided a mechanism for measuring performance, and showed a glimpse into the overhead imposed by the first version of the Dashboard. For our case study the overhead is acceptable, but for performance critical applications, it might be too large. 

    One of the downsides of the current implementation is the fact that the dashboard runs in the same process as the main application and that Python only supports green threads. The performance could be greatly improved if instead the dashboard would run in a different process, %and the information would be posted to it by using a messaging service.
    and a message queue was used to allow the decoupling of the monitoring itself from the persistence of measurements to the dashboard DB. 
    However, that would complicate the installation and configuration of the \tool.
    %, and for this kind of architecture, there are also other alternatives.


  \subsection*{Limitations of the Measurements}
    One limitation of the benchmarking we used in Section~\ref{sec:overhead} is that it does not use a more realistic mix of workload, but rather, measures individual endpoints. It might be that multiple concurrent and diverse endpoints would result into different performance impact. In fact, based on the data in the dashboard one could actually construct a statistically relevant workload mix. At this point, this constitutes future work.


  \subsection*{Lack of Empirical Proof of Ease of Adoptability}

    We designed the dashboard to make it easy to adopt and we think this is to a certain degree apparent. However, we have only presented a single use case where the API developers have used the dashboard. It would be worthwhile to organize a dedicated study to observe the challenges and opportunities of the adoption of such a dashboard across different applications. It will also be important to see whether the perspectives presented here are relevant for other developers, and also which other perspectives are missing.


  % \subsection*{Limitation of Visualizations }

  %   The visualizations for the user experiene perspectives as presented in Section \ref{sec:user} have been tested with several hundred users (of which about \activeUserCount were active during the course of the study), but the scalability of the visualizations must be further investigated for web services with tens of thousands of users.


  \subsection*{Lack of Flexibility in Allowing Multiple Groupings}

    To be agile, a tool has to be easily adaptable to different usage scenarios. However, currently one of the current limitations of \tool is that that one can specify only one type of grouping. This is insufficient for cases where multiple groupings are desirable. In the \zee case study this need emerged in the context in which it would be good to grouping the requests also by the client application that sends the request. This feature is currently under development.

  \subsection*{Limitations to Evolution Monitoring}

      The advantage of the presented approach to evolution monitoring based on observing the \code{.git} folder is the need for minimal configuration effort, as discussed in the presentation of the \tool. The disadvantage is that it will consider on equal ground the smallest of commits, even one that modifies a comment, and the shortest lived of commits, e.g.~a commit which was active only for a half an hour before a new version with a bug fix was deployed, with major and minor releases of the software. %as a distinct way of grouping the data points. 
    A mechanism to control which versions are important for monitoring purposes is therefore required to be added to the \tool.


  \subsection*{Endpoint Provenance }

    The \tool can not detect endpoint renames, and thus the history is normally lost in the case of an API rename. The problem can be approached in a similar way to the {\em provenance problem} in source code evolution analysis)\cite{Davi11a} by inferring that a newly introduced endpoint with similar patterns of request traffic might be the same as one that is not used anymore. Since this would still be a statistical approach, the integration in the tool would have to ask the user for confirmation. 
    However, this would work better in a system where endpoint tracking is activated by default for all the endpoints. Otherwise, it would require the user to remember to always enable the tracking of the new endpoint. 


  % \subsection*{Monitoring Failures} 

  %   A Special Type of Outlier: Exceptions. TODO: probably something for the journal extension ...statistics about which endpoints fail most often might be useful too.
  %   same information as the outlier maybe?

  %   - tracking other types of information: exceptions, etc. especially since logs usually get lost... nobody has time to look in them...


  \subsection*{Performance Prediction}

	In Section~\ref{sec:testing} we provided evidence towards our assumption that integration testing can be used for performance triangulation across service versions. The idea is to use a synthetic load based on unit testing to drive endpoints before they are deployed in production and observe whether performance improves or deteriorates on a version by version basis. The next logical step is to use this information for performance prediction purposes. A full blown performance prediction feature would require advanced statistical work, but it can be used as an advanced warning system for the overall performance trend for a given set of endpoints. As discussed in Section~\ref{sec:testing} however further evaluation of this feature is currently ongoing.

  \subsection*{Distributed Deployment}

	Last but not least, the presented approach and case study assume that the monitored Flask application is deployed as a single node, i.e. without support for horizontal scaling by means of replication~\cite{vaquero2011dynamically}  --- or at least that the application owner is interested in monitoring each application deployment in isolation. This is a clear limitation of this work which we are already working towards addressing by allowing multiple application instances to be registered with the same dashboard, essentially providing a ``meta-dashboard'' approach through which application endpoint performance and utilization can be monitored and visualized in a unified way. Due to the addition of an orthogonal to the service evolution dimension that the \tool currently supports, this requires a significant effort of re-engineering on our part.   
	
	
%	-   Supporting situations where the API is deployed across multiple containers for example.
%	\ml{@V: Do you want to add this? Or can we skip it?}

%    - threats to the validity of the constant-load performance testing. performance regressions? 
%
%
%    - time as measured in days and commits are two sides of the same coin. 
%
%    \ml{@V: do you want to give a first stab to this?}


  % \subsection*{Advanced Outlier Analysis}

  %   - We need more advanced outlier analysis tools... Instead of manually searching in the page to build stats about Google vs. Microsoft

  %   - outlier detection and monitoring is important. manual exploration is critical. however, one needs a performant way of analyzing multiple logs. in the example we saw we had to use the search facility to investigate our hypothesis that Google Translate was slower than Microsoft Translate. Moreover, these trends might change too... 





% - the endpoint where the unit-testing results can be uploaded is prone to being tampered with by Bob who would want to upload junk information to mess up with the dashboard. to solve this, a UUID can also be added that is generated by the server, and thus prevents anybody else but the rightful owner of the dashboard 

% - user management. although not explained in the paper, the dashboard also comes with a user management system. and two classes of users. admins and guests.


% - detecting minor versions which don't need to be tracked, since they didn't touch the performance of the system. E.g. a modification of the README, etc. 

% - discuss later one situation in which opt-out might be desirable. 

% - branches. the system does not care. it links to whatever commit is the current one. it could be that visualizations could be developed to comapre branches, but this optiona. future work. 



  










%





%!TEX root=restructured.tex

\section{Related Work}
\label{sec:related}

\todo{Refer the reader to~\cite{ghezzi2007run} and ~\cite{metzger2010analytical} for a more extensive discussion}

There is a long tradition of using visualization for gaining insight into software performance. Tools like Jinsight \cite{Pauw02a} and Web Services Navigator \cite{Pauw05} pioneered such an approach for Java and for Web Services that communicate with SOAP messages. Both have an ``omniscient'' view of the services / objects and their interactions. As opposed to them, in our work we present an analytics platform which focuses on monitoring a single Python web service from its own point of view.

From the perspective of service monitoring, our work falls within the server-side run-time monitoring of services ~\cite{ghezzi2007run}. While we don't implement the more advanced features of related monitoring solutions like QoS policies driving the monitoring, it presents nevertheless an easy to use approach support improving the performance of web applications. 

% \va{Mircea: Consider removing the rest for space...}
% An existing monitoring tool is Pingdom \footnote{https://www.pingdom.com/company/why-pingdom}, which monitors the uptime of an existing web-service. This tool works by pinging the websites (up to 60 times) every minute automatically. Thus this creates a lot of overhead and is bound to be noisy since it will also be influenced by the speed of the network connection\footnote{Another problem is that such a tool would }

% \todo{Runscope? Others?}


\section{Conclusion and Future Work}
\label{sec:conclusions}

In the previous sections we discussed our proposal for a monitoring dashboard for Flask-based web applications as the means for providing a low effort and flexible analytics solution to projects on a budget. The emphasis is in allowing application developers to gain insight into how the performance and utilization of their services evolves together with the application itself. For this purpose the \tool allows for integration with both \git and the Travis CI platform. We also discussed more advanced features of the dashboard in the form of grouping results by user groups, and provided evidence towards the use of unit testing-sourced synthetic loads as a potential limited form of performance prediction. Finally we measured the overhead of the proposed solution within acceptable by the case study application owner limits. Section~\ref{sec:discussion} which discusses what we perceive as limitations of our proposal and also potential challenges for other similar tools, effectively acts also as a roadmap for further development, with the ultimate goal of increasing the adoption of the \tool by as many projects as possible. This will allow us to evaluate our proposal in depth in the future.

% Some of the tasks have already been identified as ongoing work (allowing different types of grouping, enabling performance prediction, developing a meta-dashboard) while others constitute short- to medium-term future work (identification of important versions to be monitored, support for endpoint provenance) 

%\todo{update as appropriate}
%
%In this paper we have shown that it is possible to create a monitoring solution which provides basic insight into web service utilization and performance  with very little effort from the developer. The user group that we are aiming for with this work is application developers using Flask and Python to build web applications with limited or no budget for implementing their own monitoring solutions. The emphasis is in allowing such users to gain insight into how the performance of the service evolves together with the application itself. We believe that the same architecture, and lessons can be applied to other frameworks and other languages.
%
%In the future, we plan to perform case studies with other sytstems, with the goal of discover other needs and to wean out the less useful visualizations in the \tool. We plan to also extend the tool towards supporting multiple deployments of the same applications across multiple nodes (e.g. for the situations where the application is deployed together with a load balancer). Finally, we plan to integrate \tool with unit testing as a complementary source of information about performance evolution.



% 
\appendix

  \subsection{Changing the Default /dashboard Endpoint }

  THIS STUFF SHOULD BE MOVED TO THE APPENDIX. And in a footnote we just mention that ... \ins{REQUIREMENT}: The system must provide an easy way to add more specific configurations if needed. A way of overwriting the common sense defaults.

  Further configuration is not needed possible by adding additional statements before the binding definition, for example,

  \begin{lstlisting}[style=custompython]
  ...
  dashboard.config.link = 'mydashboard'
  dashboard.bind(flask_app)
  \end{lstlisting}
  
  allows for a custom route (\code{/mydashboard}) to the dashboard to be defined by the programmer. An external configuration file can be used instead, or in addition to this:
  
  \begin{lstlisting}[style=custompython]
  ...
  dashboard.config.from_file('config.cfg')
  ...
  \end{lstlisting}


  \subsection{For those who don't use GIT}


    \textit{Manual} version control requires the developer to tag each new version of the application with an appropriate version identifier~\cite{papazoglou2011managing} using the \code{APP\_VERSION} configuration parameter, for example by adding to the configuration file:
    
      \begin{lstlisting}[style=custompython]
      [dashboard]
      APP_VERSION=<versionID>
      \end{lstlisting}





% references section

\bibliographystyle{abbrv}
\bibliography{paper}


% that's all folks
\end{document}


